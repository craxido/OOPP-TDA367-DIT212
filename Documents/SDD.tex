\documentclass{article}
\usepackage{graphicx}
\usepackage[utf8]{inputenc}
\usepackage{caption}

\setlength\fboxsep{0pt}
\setlength\fboxrule{0.5pt}

\title{
    System design document for group 16
    \author{Erik Sjöström,
            Filip Labe,
            Jonatan Källman,
            Sarosh J. Nasir}
    \date{\today \\v1.0}         
}

\begin{document}
\maketitle

\section{Introduction}

\subsection{Design Goals}

Monster Clicker is a mobile game for Android. So the program must therfore run on Android, it must be able to get input from the user, and display the result of these inputs. 

\subsection{Definitions, acronyms, abbriviations}
\begin{itemize}
    \item \emph{Monster Clicker} - The name of the project.
\end{itemize}

\section{System architecture}

Monster Clicker starts when the user opens the application. From there the user has
the option of visiting several different activities. Each activity presents and
represents different use cases such as, buying upgrades, viewing the map, attacking monsters, etc. \\ \noindent
Monster Clicker ends when the user either exits the app or closes it using the android application manager.

\subsection{Programming patterns}
Monster Clicker, just like any other piece of software relies on different design/architecture patterns, such as:
\begin{itemize}
    \item Game Loop \cite{game-loop}
    \item Model View Presenter \cite{MVP}
    \item Singelton-pattern
    \item Factory-pattern
    \item Observer-pattern
\end{itemize}

\subsection{Dependencies}
Monster Clicker is a self contained piece of software, whch means that it does not
depend on anything but itself. The application contains all the information 
required to be able to run. \\

\noindent
The application is subdivided into several packages:
\begin{itemize}
    \item \emph{Player} 
    \item \emph{Monster Pack}
    \item \emph{Map}
    \item \emph{Stats}
    \item \emph{Home}
    \item \emph{Shop}
    \item \emph{Upgrade}
    \item \emph{Clock}
\end{itemize}

\noindent
Gem Clicker is developed for the Android operating system and is guaranteed to
run on android phones running 4.0 or later versions of the operating system.  

\section{Subsystem decomposition}

\subsection{Player}
This package contains a player model, and an interface for this model. The model contains the state of the player, damage, gold, etc. And methods to access and/or change the state. The interface provides a layer between the model and anyone who wishes to access it, only exposing non-internal methods.
\newpage
\subsubsection{Diagrams}
\begin{center}
    Package UML diagram: \\
    \fbox{ \includegraphics[scale=0.3]{uml/playerUml.png}}
    Sequence diagram of getState()\\
    \fbox{ \includegraphics[scale=0.2]{seqDiagrams/playerModel.png}}
\end{center}

\subsubsection{Quality}
The tests for \emph{PlayerModel} can be found in \emph{/OOPP/app/src/test/\ldots/oopp/PlayerModelTest.java}

\subsection{Monster Pack}
This package contains a monster model, and a monster factory. The model contains the state of the monster, gold, health, etc. A constructor to set the state and methods
to modify the state. The factory handles the creation of monsters in a simplified way.

\subsection{Diagrams}
\begin{center}
    Sequence diagram of what happens when a monster is clicked.\\
    % \fbox{\includegraphics[scale=0.2]{seqDiagrams/monsterClicked.png}}
\end{center}

\subsubsection{Quality}

\subsection{Map}
The package which represents the map activity in the app. From here the player can change area. Takes care of the area, level and map instantiation.
The map package consists of the files:
\begin{itemize}
    \item Area: Container class for Levels.
    \item Level: A class which contains information about the monster and which methods can be used to act on the monster. Plays a significant role as it contains the monster which you are fighting in the main activity.
    \item Map: The model.
    \item MapActivity: The view.
    \item MapPresenter: The presenter.
    \item areaType: A simple enum type.
    \item levelFactory: Produces levels.
    \item MapMVPInterface: The interfaces which regulate how the classes communicate internally and how other packages can communicate with the the presenter.
\end{itemize}

\subsection{Stats}
The stats package presents the state stored in the PlayerModel in a readable way.\\
The stats package consists of the files:
\begin{itemize}
    \item StatsActivity: The view.
    \item StatsPresenter: The presenter.
\end{itemize}

\subsubsection{Diagrams}
\begin{center}
    Package UML diagram:\\
    \fbox{ \includegraphics[scale=0.3]{uml/statsUml.png}}
\end{center}

\subsection{Home}
The home package represents the home activity in the app. It consists of the files:
\begin{itemize}
    \item Home: The model
    \item HomePresenter: The presenter
    \item HomeActivity: The view
    \item HomeMVPInterface: The interface which regulates how the presenter communicates with the model and the view, and how everyone else can communicate with the presenter.
\end{itemize}

\subsection{Shop}
The shop package represents the shop activity in the app. It consits of the files:
\begin{itemize}
    \item Shop: The model
    \item ShopPresenter: The presenter
    \item ShopActivity: The view
    \item ShopMVPInterface: The interface which regulates how the presenter communicates with the model and the view, and how everyone else can communicate with the presenter.
\end{itemize}

\subsection{Upgrade}
Contains the class Upgrade which is the class representation of an upgrade in this game, used in Shop and Home, where one can buy upgrades.

\subsubsection{Diagrams}
\begin{center}
    Package UML diagram:
    \includegraphics[scale=0.5]{uml/upgradeUml.png}
\end{center}

\subsubsection{Quality}

\subsection{Clock}
The clock package is the implementation of the game-loop. The Runner class is implemented as a singelton so that everone that wants to register to the loop can get an instance of the runner. 

\subsubsection{Diagrams}

\subsubsection{Quality}

\section{Persistent data management}
Gem Clicker saves the state of the app when the app is closed. 
The data is represented as a simple key-value storage, which is handled by android internaly.

\section{Access control and security}
There are no different roles for using this application. The only role is that of the user,
and the only permission required of the user is to use the storage space of the phone.

\section{References}
\begin{thebibliography}{9}
    
    \bibitem{game-loop}
        http://gameprogrammingpatterns.com/game-loop.html
    
    \bibitem{MVP}
        https://en.wikipedia.org/wiki/Model–view–presenter

\end{thebibliography}

\section{Appendix}
\subsection{Package structure}
\begin{center}
    \fbox{ \includegraphics[scale=0.6]{uml/packageStructure.png}}
\end{center}

\end{document}
